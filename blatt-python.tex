\documentclass{blatt}
\let\raggedsection\centering

\usepackage{mathpazo}

\begin{document}

\maketitle{Python, eine moderne Programmiersprache}


\section{Was ist Python?}

Python ist eine moderne Programmiersprache, die sich bei vielen Entwicklerinnen
und Entwicklern großer Beliebtheit erfreut. Sie wurde Anfang der 90er Jahre von
Guido van Rossum, einem niederländischen Programmierer, entworfen und wird seitdem
von einer großem Team Freiwilliger als freies Open-Source-Projekt gepflegt.

Python ist eine so genannte höhere Programmiersprache, die die Zeit der
Programmiererin oder des Programmierers über die Zeit des Computers stellt.
Gelegentlich ist Python-Code also etwas langsamer als zum Beispiel mühsam
handoptimierter C-Code -- dafür lässt es sich in Python viel schneller und
angenehmer entwickeln.

Zu den Anwendungsbereichen von Python gehören unter anderen die Web-, System-
und Spiele-Entwicklung. Außerdem wird Python für wissenschaftliche Zwecke
eingesetzt -- das ist der Aspekt, den wir beleuchten werden.


\section{Installation von Python}

Damit ein Computer Python-Programme ausführen kann, muss man zunächst den
Python-Interpreter installieren. Das ist die erste Aufgabe an euch!

\paragraph{Unter Ubuntu Linux und anderen Debian-Derivaten.}
Öffne eine Konsole und setz den Befehl \texttt{sudo apt-get install
python-numpy python-scipy python-matplotlib} ab. Das war's schon.

\paragraph{Unter Mac OS X und Windows.} Lade auf
\url{http://continuum.io/downloads} das Komplettpaket Anaconda herunter und
klicke dich durch die Installation. Wähle die Python-Version~2.7 und unter
Windows die 32-Bit-Version (auch, wenn dein Computer ein 64-Bit-Prozessor haben
sollte).


\section{Erste Schritte}

\begin{itemize}
\item Hallo Welt
\item Einfache Rechnungen
\item Grundtechniken im Debugging
\end{itemize}


\section{Plotten}


\section{Newton-Verfahren}

\end{document}
