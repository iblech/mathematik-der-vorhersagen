\documentclass{blatt}
\let\raggedsection\centering

\usepackage{mathpazo}

\begin{document}

\maketitle{Python, eine moderne Programmiersprache}


\section{Was ist Python?}

Python ist eine moderne Programmiersprache, die sich bei vielen Entwicklerinnen
und Entwicklern großer Beliebtheit erfreut. Sie wurde Anfang der 90er Jahre von
Guido van Rossum, einem niederländischen Programmierer, entworfen und wird seitdem
von einer großem Team Freiwilliger als freies Open-Source-Projekt gepflegt.

Python ist eine so genannte höhere Programmiersprache, die die Zeit der
Programmiererin oder des Programmierers über die Zeit des Computers stellt.
Gelegentlich ist Python-Code also etwas langsamer als zum Beispiel mühsam
handoptimierter C-Code -- dafür lässt es sich in Python viel schneller und
angenehmer entwickeln.

Zu den Anwendungsbereichen von Python gehören unter anderen die Web-, System-
und Spiele-Entwicklung. Außerdem wird Python für wissenschaftliche Zwecke
eingesetzt -- das ist der Aspekt, den wir beleuchten werden.


\section{Installation von Python}

Damit ein Computer Python-Programme ausführen kann, muss man zunächst einen
\emph{Python-Interpreter} installieren. Das ist die erste Aufgabe an euch!

\paragraph{Unter Ubuntu Linux und anderen Debian-Derivaten.}
Öffne eine Konsole und setz den Befehl \texttt{sudo apt-get install
python-numpy python-scipy python-matplotlib} ab. Das war's schon.

\paragraph{Unter Mac OS X und Windows.} Lade auf
\url{http://continuum.io/downloads} das Komplettpaket Anaconda herunter und
klicke dich durch die Installation. Wähle die Python-Version~2.7 und unter
Windows die 32-Bit-Version (auch, wenn dein Computer ein 64-Bit-Prozessor haben
sollte).


\section{Erste Schritte}

\subsection{Hallo, Welt!}

Das erste Programm, dass man schreibt, wenn man eine neue Programmiersprache
lernt, ist \emph{Hallo Welt!}: Ein Programm, dass eine kurze Meldung ausgibt
und sich dann beendet. In Python sieht das so aus:

% XXX: Syntax Highlighting
\begin{verbatim}
    #!/usr/bin/env python
    # -*- coding: utf-8 -*-

    print("Hallo Welt!")
\end{verbatim}

Tippe das Programm ab (oder kopiere den Quellcode von XXX) und führe es aus.
Wenn es nicht klappt, dann melde dich bei uns (XXX Forum?).

Die ersten beiden Zeilen finden sich in jedem Python-Programm. Die erste ist
vor allem unter Linux und OS~X wichtig; sie ist der Indikator für das
Betriebssystem, dass es sich um ein Python-Programm handelt. Es ist guter Stil,
sie auch unter Windows beizubehalten. Zeile~2 hat technische
Gründe.\footnote{Zeile~2 setzt fest, dass der Programmcode in der
Zeichenkodierung UTF-8 (und nicht etwa in dem älteren Standard ISO-8859-1)
interpretiert werden soll. Eine Zeichenkodierung gibt an, wie Umlaute und
andere Zeichen, die über den Sprachschatz des amerikanischen ASCII-Standards
aus den 70er Jahren (XXX) hinausgehen, als Bytes gespeichert werden sollen.}

Zeile 3 ist eine Leerzeile. Leerzeilen haben für Python selbst keine Bedeutung,
können also nach Belieben hinzugefügt oder entfernt werden. Es ist aber guter
Stil, einzelne Sinneinheiten durch Leerzeilen zu trennen, um den Code für den
Menschen übersichtlicher zu gestalten. Es gibt ja auch einen Grund, wieso es im
Deutschen und vielen anderen natürlichen Sprachen Absätze gibt.

Die eigentliche Arbeit wird durch Zeile~4 angestoßen. Dort wird die in Python
vordefinierte Prozedur \texttt{print} mit dem \emph{Argument} \texttt{\"{}Hallo,
Welt!\"} aufgerufen. Die Schreibweise soll an die in der Mathematik übliche
Notation für Funktionen erinnern -- dort schreibt man zum Beispiel
"`$\sin(5)$"'.


\subsection{Die interaktive Python-Shell}

\begin{itemize}
\item Einfache Rechnungen
\end{itemize}


\subsection{Programmierfehler und wie man mit ihnen umgeht}

Beim Programmieren macht man Fehler. Von denen gibt es vor allem zwei Arten:
\emph{Syntaxfehler} und \emph{inhaltliche Fehler}. Ein Syntaxfehler tritt auf,
wenn man sich nicht an die Schreibregeln von Python hält. Zum Beispiel wird der
Code
\begin{verbatim}
    print("Hallo, Welt!"
\end{verbatim}
nicht funktionieren, da die schließende Klammer fehlt. Syntaxfehler werden vom
Python-Interpreter direkt nach dem Start, noch bevor er mit dem Ausführen des
Codes beginnt, erkannt und gemeldet.

Programmiersprachen wie Python sind in ihrer Syntax sehr streng. XXX Beispiel,
XXX Warnung vor falscher Zeilenangabe im Fehlerbericht, XXX Schleichwerbung für
statische Typsysteme

\begin{itemize}
\item inhaltliche Fehler (Beispiel: Variablenvertauschung)
\item Umlaute
\item Grundtechniken im Debugging
\end{itemize}


\section{Plotten}


\section{Newton-Verfahren}

\end{document}
