\documentclass[12pt,compress,ngerman,utf8,t]{beamer}
\usepackage[ngerman]{babel}
\usepackage{calc}
\usepackage{ragged2e,wasysym,multicol,mathtools}
\usepackage[protrusion=true,expansion=true]{microtype}
\usepackage{tikz}
\hypersetup{colorlinks=true}

\graphicspath{{images/}}

\title{\large Wie vollbringen künstliche Intelligenzen das Kunststück des Lernens?}
\author[Ingo Blechschmidt]{\textcolor{white}{Ingo Blechschmidt \\ mit Dank an XXX}}
\date[2017-04-22]{\vspace*{-5em}\ \\\textcolor{white}{\scriptsize Institut für Mathematik \\ Universität Augsburg \\ 22. April 2017 \\}}

%\usetheme{Warsaw}
\useinnertheme[shadow=true]{rounded}
\useoutertheme{split}
\usecolortheme{orchid}
\usecolortheme{whale}
\setbeamerfont{block title}{size={}}

\useinnertheme{rectangles}

\usecolortheme{seahorse}
\definecolor{mypurple}{RGB}{150,0,255}
\setbeamercolor{structure}{fg=mypurple}
\definecolor{myred}{RGB}{150,0,0}
\setbeamercolor*{title}{bg=myred,fg=white}
\setbeamercolor*{titlelike}{bg=myred,fg=white}

\usefonttheme{serif}
\usepackage[T1]{fontenc}
\usepackage{libertine}

\renewcommand{\_}{\mathpunct{.}\,}
\newcommand{\BB}{\mathbb{B}}
\newcommand{\M}{\mathcal{M}}
\newcommand{\R}{\mathrm{R}}
\newcommand{\NN}{\mathbb{N}}
\newcommand{\RR}{\mathbb{R}}

\setbeamertemplate{navigation symbols}{}

\setbeamertemplate{title page}[default][colsep=-1bp,rounded=false,shadow=false]
\setbeamertemplate{frametitle}[default][colsep=-2bp,rounded=false,shadow=false,center]

\newcommand{\hil}[1]{{\usebeamercolor[fg]{item}{\textbf{#1}}}}
\setbeamertemplate{frametitle}{%
  \vskip1em%
  \leavevmode%
  \begin{beamercolorbox}[dp=1ex,center]{}%
      \usebeamercolor[fg]{item}{\textbf{\textsf{\Large \insertframetitle}}}
  \end{beamercolorbox}%
}

\setbeamertemplate{footline}{%
  \leavevmode%
  \hfill%
  \begin{beamercolorbox}[ht=2.25ex,dp=1ex,right]{}%
    \usebeamerfont{date in head/foot}
    \insertframenumber\,/\,\inserttotalframenumber\hspace*{1ex}
  \end{beamercolorbox}%
  \vskip0pt%
}

\newcommand{\backupstart}{
  \newcounter{framenumberpreappendix}
  \setcounter{framenumberpreappendix}{\value{framenumber}}
}
\newcommand{\backupend}{
  \addtocounter{framenumberpreappendix}{-\value{framenumber}}
  \addtocounter{framenumber}{\value{framenumberpreappendix}}
}

\setbeameroption{show notes}
\setbeamertemplate{note page}[plain]

\newcommand{\imgslide}[3]{{\usebackgroundtemplate{\parbox[c][\paperheight][c]{\paperwidth}{\centering\includegraphics[width=\paperwidth]{#1}}}\begin{frame}[plain,b]\tiny Quelle: \href{#2}{#3}\par\end{frame}}}

\newcommand{\portrait}[4]{\begin{column}{#3\textwidth}\centering\includegraphics[height=#4\textheight]{#1}\\{\scriptsize #2\par}\end{column}}

\begin{document}

% https://static2.gamespot.com/uploads/original/1557/15576725/2944861-hogwarts.jpg
{\usebackgroundtemplate{\includegraphics[height=\paperheight]{hogwarts}}
\frame{\vspace*{9em}\titlepage}}
\frame{\tableofcontents}

% Einführungsvideos
% * https://www.youtube.com/watch?v=MzJ0CytAsec
%   Windows Vista Speech Recognition Tested - Perl Scripting
% * https://www.youtube.com/watch?v=M1ONXea0mXg
%   Hound Internal Demo

\section{Erfolge von KI}

\begin{frame}
  \centering
  \bigskip\bigskip

  \Huge \hil{Teil I}

  \bigskip
  \Large\textbf{Jüngste Erfolge von \\ künstlicher Intelligenz}
  \par

  \vfill
  \vfill
  \vfill
  \begin{columns}
    % http://nerdist.com/wp-content/uploads/2016/03/DeepMind-Sedol-Go-Match-Feature-Image-03082016.jpg
    \portrait{wavenet}{\href{https://deepmind.com/blog/wavenet-generative-model-raw-audio/}{Sprachsynthese}}{0.17}{0.25}
    \portrait{deepmind-match}{\href{https://de.wikipedia.org/wiki/AlphaGo}{AlphaGo}}{0.25}{0.25}
    \portrait{neural-style}{\href{https://github.com/jcjohnson/neural-style}{Stiltransfer}}{0.25}{0.25}
    \portrait{magenta-jam-session}{\href{https://magenta.tensorflow.org/blog/2016/12/16/nips-demo/}{Jammen mit Magenta}}{0.25}{0.25}
  \end{columns}
\end{frame}

% Für Wavenet abspielen: resources/speaker-*.wav
% Für Stiltransfer: resources/jcjohnson*.html
% Für die Supervergrößerung: resources/neural-enhance.gif


\section[Funktionsweise]{Funktionsweise künstlicher neuronaler Netzwerke}

{\usebackgroundtemplate{\includegraphics[height=\paperheight]{neuron-art}}
\begin{frame}
  \centering
  \bigskip\bigskip

  \Huge \hil{Teil II}

  \bigskip
  \Large\textbf{Funktionsweise künstlicher neuronaler Netzwerke}
  \par

  \vfill\small
  \begin{enumerate}
    \item Aufbau eines einfachen Netzes
    \item Bewertung durch eine Kostenfunktion
    \item Fehlerminimierung mittels Gradientenabstieg
  \end{enumerate}
\end{frame}}


\subsection{Netzaufbau}

% Vorlage von Kjell Magne Fauske, http://www.texample.net/tikz/examples/neural-network/
\begin{frame}{Aufbau eines einfachen Netzes}
  \def\layersep{2cm}
  \begin{tikzpicture}[shorten >=1pt,->,draw=black!50, node distance=\layersep]
    \tikzstyle{every pin edge}=[<-,shorten <=1pt]
    \tikzstyle{every node}=[font={\small}]
    \tikzstyle{neuron}=[circle,fill=black!25,minimum size=17pt,inner sep=0pt]
    \tikzstyle{input neuron}=[neuron, fill=green!80];
    \tikzstyle{output neuron}=[neuron, fill=red!50];
    \tikzstyle{hidden neuron}=[neuron, fill=blue!50];
    \tikzstyle{annot} = [text width=4em, text centered]

    \node[input neuron, pin=left:Eingabe 1] (I-1) at (0,-1.2*1) {\only<3->{$0{,}1$}};
    \node[input neuron, pin=left:Eingabe 2] (I-2) at (0,-1.2*2) {\only<3->{$0{,}7$}};
    \node[input neuron, pin=left:Eingabe 3] (I-3) at (0,-1.2*3) {\only<3->{$0{,}2$}};
    \node[input neuron, pin=left:Eingabe 4] (I-4) at (0,-1.2*4) {\only<3->{$0{,}4$}};

    \foreach \name / \y in {1,...,5}
      \path[yshift=0.5cm]
        node[hidden neuron] (H-1-\name) at (\layersep,-1.2*\y cm) {};
    \foreach \name / \y in {1,...,5}
      \path[yshift=0.5cm]
        node[hidden neuron] (H-2-\name) at (2*\layersep,-1.2*\y cm) {};

    \only<5->{
      \node at (H-1-2) {$y$};
    }

    \node[output neuron,pin={[pin edge={->}]right:Ausgabe 1}, right of=H-2-2] (O-1) {};
    \node[output neuron,pin={[pin edge={->}]right:Ausgabe 2}, right of=H-2-4] (O-2) {};

    \only<1>{
      \foreach \source in {1,...,4}
        \foreach \dest in {1,...,5}
          \path (I-\source) edge (H-1-\dest);
    }
    \only<2->{
      \foreach \source in {1,...,4}
        \foreach \dest in {2}
          \path (I-\source) edge (H-1-\dest);
    }
    \only<4>{
      \path (I-1) -- (H-1-2) node[midway] {3};
      \path (I-2) -- (H-1-2) node[midway] {4};
      \path (I-3) -- (H-1-2) node[midway] {1};
      \path (I-4) -- (H-1-2) node[midway] {5};
    }
    \only<1>{
      \foreach \source in {1,...,5}
        \foreach \dest in {1,...,5}
          \path (H-1-\source) edge (H-2-\dest);
    }
    \only<2->{
      \foreach \source in {2}
        \foreach \dest in {1,...,5}
          \path (H-1-\source) edge (H-2-\dest);
    }

    \foreach \source in {1,...,5}
      \path (H-2-\source) edge (O-1);
    \foreach \source in {1,...,5}
      \path (H-2-\source) edge (O-2);

    \node[annot,above of=H-1-1, node distance=1cm] (hl1) {Verborgene Schicht};
    \node[annot,above of=H-2-1, node distance=1cm] (hl2) {Verborgene Schicht};
    \node[annot,left of=hl1] {Eingabe\-schicht};
    \node[annot,right of=hl2] {Ausgabe\-schicht};

    \only<5->{
      \node[below of=O-2] (y) {\scriptsize $y = \sigma(0{,}1\cdot3 + 0{,}7\cdot4 + 0{,}2\cdot1 + 0{,}4\cdot5)$};
      \draw[<-, in=120] (H-1-2.east)++(-0.1cm,0cm) to (y.west);
    }
  \end{tikzpicture}
\end{frame}

\end{document}
